\documentclass[12pt,journal]{article}
\hyphenation{op-tical net-works semi-conduc-tor}

\usepackage{url}
\usepackage[hidelinks]{hyperref}
\usepackage[backend=biber,style=ieee]{biblatex}
\addbibresource{Testing.bib}
\usepackage{geometry}
\usepackage{fancyhdr}
\usepackage{afterpage}
\usepackage{graphicx}
\usepackage{amsmath,amssymb,amsbsy}
\usepackage{pdflscape}
\usepackage{tikz}
\def\checkmark{\tikz\fill[scale=0.4](0,.35) -- (.25,0) -- (1,.7) -- (.25,.15) -- cycle;} 
\usepackage[activate={true,nocompatibility},final,tracking=true,kerning=true,spacing=true,factor=1100,stretch=10,shrink=10]{microtype}
% activate={true,nocompatibility} - activate protrusion and expansion
% final - enable microtype; use "draft" to disable
% tracking=true, kerning=true, spacing=true - activate these techniques
% factor=1100 - add 10% to the protrusion amount (default is 1000)
% stretch=10, shrink=10 - reduce stretchability/shrinkability (default is 20/20)
\usepackage{dcolumn,array}
\usepackage{tocloft}
\usepackage[section]{placeins}
\usepackage[english]{babel}
\usepackage{todonotes}
\usepackage{blindtext}
\usepackage{amsthm}
\usepackage{setspace}
\usepackage[babel=true]{csquotes}
\blindmathtrue
\usepackage[acronym]{glossaries}
\usepackage[section]{algorithm}
\usepackage{algpseudocode}
\usepackage{listings}
\newtheorem{mydef}{Definition}

\begin{document}
\doublespace
\title{Design of Experiments \\ CSE 565 Assignment \#2}
\author{Jeremy Wright - 1000738685}

% make the title area
\maketitle

\section{Testing a Tone Designer App}
Design of Experiments lets us cover pairwise test cases more efficiently than
trying to deal with combinatoric explosion directly. For this example we will
design a digital tone application that simulates the sound of professional audio
equipment. This allows one to inexpensively sound like a professional recording
artist without requiring all the professional gear.  In this case we will
evaluate the impact of string gauge, guitar pick thickness, guitar model,
amplifier model and finally the physical speaker used to produce the sound. Our
inputs are described by Table~\ref{tab:inputs}.

\begin{table}
    \centering
    \caption{Inputs for a guitar modeling app.}
    \label{tab:inputs}
    \begin{tabular}{c | c | c | c | c }
        \hline
        Output Source & Amp Model & Guitar  & Pick Size  &  String Gauge \\
        \hline \hline 
  Tube Amp                          & Fender Twin & Stratocaster    & 0.2 mm & 8\\
  Digital Amp                       & Boogie      & Telecaster      & 0.3 mm & 9\\
  Flat-Response-Full-Range          & Marshall 15 & Martin          & 0.4 mm & 10\\
                                    & Marshall 30 & 12 string       & 0.5 mm & 11\\
                                    & Marshall 75 & Mustang         & 0.8 mm & 12\\
                                    & Vox 15      & Ibanez          & 1 mm   & \\
                                    & Vox 30      & Paul Reed Smith & 1.5 mm & \\
                                    & Vox 75      & Squire          & 2 mm   & \\
                            
        \hline
    \end{tabular}
\end{table}

Evaluating pairwise test cases manually can be a time consuming job, instead
we can use a tool such as Microsoft's PICT \autocite{czerwonka_pairwise_2008}.
PICT is a command-line tool that takes a text file and outputs the pairwise
test cases. Using a text file also enables us to check in and manage the test
 cases with our source code, since our version control software can easily
 understand plain text. Listing~\ref{lst:pict_input} then describes our
 modeling app.  We then run this through the tool with
 {\ttfamily c:\textbackslash Users\textbackslash > pict.exe guitar\_tone.pict
 > guitar\_tone.tests}, which generates Listing~\ref{lst:output}. The tool
 generates significantly fewer tests than the combinatoric approach requires as 
 Table~\ref{tab:comparison} demonstrates.  

\begin{table}
    \centering
    \caption{Comparison of test strategies}
    \label{tab:comparison}
    \begin{tabular}{c | r }
        \hline
Testing Method & Number of Test Cases  \\
        \hline \hline 
        Full Combinations & $3 \times 8 \times 8 \times  8 \times 5 = 7680$\\
        Minimum DOE & $8 \times 8 = 64$\\
        Microsoft PICT & 68\\
        \hline
    \end{tabular}
\end{table}

%\begin{table}
%    \centering
%    \caption{Test Cases Generated by PICT to verify our modeling app.}
%    \label{tab:test_cases}
%    \begin{tabular}{l | l | l | l | l }
%        \hline
%Output Source	& Amp Model &	Guitar &	Pick Size	& String Gauge \\
%        \hline \hline
%Digital Amp	& Boogie	& Martin	   & 1.5 mm	& 9 \\
%Digital Amp	& Boogie	& PRS	       & 1 mm	& 8 \\
%Digital Amp	& Boogie	& Strat	       & 0.5 mm	& 10 \\
%Digital Amp	& Boogie	& ibanez	   & 2 mm	& 9 \\
%Digital Amp	& Fender Twin	& PRS	   & 1.5 mm	& 8 \\
%Digital Amp	& Fender Twin	& Strat	   & 0.8 mm	& 11 \\
%Digital Amp	& Fender Twin	& Tele	   & 2 mm	& 8 \\
%Digital Amp	& Fender Twin	& mustang  & 0.5 mm	& 9 \\
%Digital Amp	& Marshall 15	& Martin   & 0.5 mm	& 10 \\
%Digital Amp	& Marshall 15	& Strat	   & 0.8 mm	& 11 \\
%Digital Amp	& Marshall 15	& ibanez   & 0.3 mm	& 8 \\
%Digital Amp	& Marshall 15	& mustang  & 0.4 mm	& 9 \\
%Digital Amp	& Marshall 30	& 12 string&	2 mm& 	10 \\
%Digital Amp	& Marshall 30	& Squire   & 1.5 mm	& 9 \\
%Digital Amp	& Marshall 30	& mustang  & 0.3 mm	& 9 \\
%Digital Amp	& Marshall 75	& PRS	   & 0.2 mm	& 9 \\
%Digital Amp	& Marshall 75	& mustang  & 1 mm	& 10 \\
%Digital Amp	& Vox 15 &	Squire	       & 0.4 mm	& 8 \\
%Digital Amp	& Vox 15 &	mustang	       & 2 mm	& 10 \\
%Digital Amp	& Vox 30 &	PRS	           & 0.8 mm	& 8 \\
%Digital Amp	& Vox 30 &	ibanez	       & 0.2 mm	& 12 \\
%Digital Amp	& Vox 75 &	mustang	       & 1.5 mm	& 12 \\
%Flat-Response-Full-Range	& Boogie	& 12 string	       & 1.5 mm	& 10 \\
%Flat-Response-Full-Range	& Boogie	& Squire	       & 0.2 mm	& 8 \\
%Flat-Response-Full-Range	& Boogie	& Tele	           & 0.3 mm	& 12 \\
%Flat-Response-Full-Range	& Fender Twin	& 12 string	   & 0.2 mm	& 10 \\
%Flat-Response-Full-Range	& Fender Twin	& Squire	   & 0.3 mm	& 10 \\
%Flat-Response-Full-Range	& Fender Twin	& ibanez	   & 0.4 mm	& 9 \\
%Flat-Response-Full-Range	& Marshall 15	& 12 string	   & 1 mm	& 11 \\
%Flat-Response-Full-Range	& Marshall 15	& PRS	       & 2 mm	& 12 \\
%Flat-Response-Full-Range	& Marshall 15	& mustang	   & 0.2 mm	& 10 \\
%Flat-Response-Full-Range	& Marshall 30	& Martin	   & 1 mm	& 8 \\
%Flat-Response-Full-Range	& Marshall 30	& ibanez	   & 0.5 mm	& 11 \\
%Flat-Response-Full-Range	& Marshall 75	& 12 string	   & 0.5 mm	& 8 \\
%Flat-Response-Full-Range	& Marshall 75	& Martin	   & 2 mm	& 9 \\
%Flat-Response-Full-Range	& Vox 15	& 12 string	       & 0.8 mm	& 9 \\
%Flat-Response-Full-Range	& Vox 15	& Martin	       & 0.2 mm	& 10 \\
%Flat-Response-Full-Range	& Vox 15	& Squire	       & 0.5 mm	& 12 \\
%Flat-Response-Full-Range	& Vox 15	& Tele	           & 1.5 mm	& 9 \\
%Flat-Response-Full-Range	& Vox 15	& ibanez	       & 1 mm	& 9 \\
%Flat-Response-Full-Range	& Vox 30	& Squire	       & 1 mm	& 11 \\
%Flat-Response-Full-Range	& Vox 30	& Strat	           & 2 mm	& 10 \\
%Flat-Response-Full-Range	& Vox 30	& mustang	       & 1.5 mm	& 8 \\
%Flat-Response-Full-Range	& Vox 75	& Martin	       & 0.3 mm	& 8 \\
%Flat-Response-Full-Range	& Vox 75	& Strat	           & 1 mm	& 9 \\
%Flat-Response-Full-Range	& Vox 75	& ibanez	       & 0.8 mm	& 10 \\
%Tube Amp	& Boogie	 & Strat	   & 0.4 mm	& 9 \\
%Tube Amp	& Boogie	 & mustang	   & 0.8 mm	& 11 \\
%Tube Amp	& Fender Twin	& Martin   & 1 mm	& 12 \\
%Tube Amp	& Marshall 15	& Squire   & 1.5 mm	& 10 \\
%Tube Amp	& Marshall 15	& Tele	   & 1 mm	& 12 \\
%Tube Amp	& Marshall 30	& Martin   & 0.8 mm	& 8 \\
%Tube Amp	& Marshall 30	& PRS	   & 0.4 mm	& 10 \\
%Tube Amp	& Marshall 30	& Strat	   & 0.2 mm	& 8 \\
%Tube Amp	& Marshall 30	& Tele	   & 0.8 mm	& 12 \\
%Tube Amp	& Marshall 75	& Squire   & 0.8 mm	& 9 \\
%Tube Amp	& Marshall 75	& Strat	   & 0.3 mm	& 12 \\
%Tube Amp	& Marshall 75	& Tele	   & 0.4 mm	& 10 \\
%Tube Amp	& Marshall 75	& ibanez   & 1.5 mm	& 11 \\
%Tube Amp	& Vox 15	& PRS	       & 0.3 mm	& 11 \\
%Tube Amp	& Vox 15	& Strat	       & 1.5 mm	& 11 \\
%Tube Amp	& Vox 30	& 12 string	   & 0.3 mm	& 9 \\
%Tube Amp	& Vox 30	& Martin	   & 0.4 mm	& 11 \\
%Tube Amp	& Vox 30	& Tele	       & 0.5 mm	& 9 \\
%Tube Amp	& Vox 75	& 12 string	   & 0.4 mm	& 12 \\
%Tube Amp	& Vox 75	& PRS          & 0.5 mm	& 8 \\
%Tube Amp	& Vox 75	& Squire       & 2 mm	& 11 \\
%Tube Amp	& Vox 75	& Tele	       & 0.2 mm	& 11 \\
%        \hline
%    \end{tabular}
%\end{table}
\section{Conclusion}
Design of Experiments allows us to significantly reduce the test effort of our
face melting guitar application while still providing good test coverage.

\clearpage
\singlespace
\lstinputlisting[basicstyle=\ttfamily,label={lst:pict_input},caption={Input
file to Microsoft's PICT program}]{guitar_tone.pict}


\lstinputlisting[basicstyle=\ttfamily,label={lst:output},caption={Generated
Test cases from PICT},numbers={left}]{guitar_tone.tests}

\clearpage
\printbibliography



\end{document}

