\documentclass[12pt,journal]{article}
\hyphenation{op-tical net-works semi-conduc-tor}

\usepackage{url}
\usepackage[hidelinks]{hyperref}
\usepackage[backend=biber,style=ieee]{biblatex}
\addbibresource{thesis.bib}
\usepackage{geometry}
\usepackage{fancyhdr}
\usepackage{afterpage}
\usepackage{graphicx}
\usepackage{amsmath,amssymb,amsbsy}
\usepackage{pdflscape}
\usepackage{tikz}
\def\checkmark{\tikz\fill[scale=0.4](0,.35) -- (.25,0) -- (1,.7) -- (.25,.15) -- cycle;} 
\usepackage[activate={true,nocompatibility},final,tracking=true,kerning=true,spacing=true,factor=1100,stretch=10,shrink=10]{microtype}
% activate={true,nocompatibility} - activate protrusion and expansion
% final - enable microtype; use "draft" to disable
% tracking=true, kerning=true, spacing=true - activate these techniques
% factor=1100 - add 10% to the protrusion amount (default is 1000)
% stretch=10, shrink=10 - reduce stretchability/shrinkability (default is 20/20)
\usepackage{dcolumn,array}
\usepackage{tocloft}
\usepackage[section]{placeins}
\usepackage[english]{babel}
\usepackage{todonotes}
\usepackage{blindtext}
\usepackage{amsthm}
\usepackage{setspace}
\usepackage[babel=true]{csquotes}
\blindmathtrue
\usepackage[acronym]{glossaries}
\usepackage[section]{algorithm}
\usepackage{algpseudocode}
\usepackage{listings}
\newtheorem{mydef}{Definition}

\begin{document}
\doublespace
\title{Unit Test Frameworks \\ CSE 565 Assignment \#1}
\author{Jeremy Wright - 1000738685}

% make the title area
\maketitle

\section{Frameworks}
Unit testing should be part of an overall test strategy to
ensure product quality. Unit testing, forms the smallest indivisible unit. Test
frameworks provide organization tools, to help the developer control the growing
mass of tests.
%\section{What is unit testing}
%Wikipedia defines unit testing as testing, ``... a unit as the smallest testable
%part of an application. \autocite{wikipedia_unit_2014}'' I posed this question
%to a colleague of mine, and his response was, ``A unit test de-tangles
%dependencies so one can test a submodules prior to merging it's functionality
%with another module.'' \autocite{brandenburg}. I like this definition since it
%attacks the heart of a problem.  There is a block of code that needs to be
%tested, it could be large or small by you are concerned with its dependencies
%prior to making the block larger. It speaks to the incremental, or even design
%motivating potential of unit tests. I feel, when used properly, unit tests can
%how motivate the design, since it forced one to think about the interface
%up-front. The unit-test by design will use the interface, hence making it easy
%to test, tends to make the module easier to understand.
%
%However, in practice unit testing can be a more visceral tool. While it does
%function as a design tool, its greatest return on investment is its ability to
%test edge conditions, and in a sense embarrassing bugs. Dr. Collofello asked in
%class, ``What if a tester wrote the unit tests instead of the developer? Would
%that be a good idea?''   My initial response was, those are my tests. Hence,
%I now use Definition~\ref{def:person_unit_test}.
%
%\begin{mydef} 
%    \label{def:person_unit_test}
%A unit test is a test to provide the developer some confidence
%a region of code works as intended before other people see the code and the
%developer is open to mockery and scandal.
%\end{mydef}

A framework is by no means necessary to perform unit testing. One method of unit
testing is to simply create separate executables for each test, and use simple
asserts to check for pre and post conditions. This method makes command line
debugging easier since each test is simply a basic executable. Test frameworks
however are quite useful when you want integration with an IDE, or continuous
integration system such as XML output for a build server. 

A unit test framework however can be quite an enabler, since it doesn't forget
to run all the tests. When not using such a framework, one has to setup some
form of automation to ensure all tests are run with every build. Build systems
such as autoconf, and CMake both support such automation, but sometimes it's
nicer to have a test tool do the work, especially when you want to generates
reports off the results.

Frameworks are typically build time dependencies, and allow developers to
annotate, or mark up their code in some manner. The test generation step then
discovers tests and generates an executable.  Frameworks also provide
organization tools called test suites which allow one to control SetUp, and
TearDown of the suite, as well as which tests get grouped together.

A favorite feature of mine I picked up from FPGA design is constrained random
test generation \autocite{constrained}. Sorting functions are ideal candidates
since there input domain is large, random test generation can expose test cases
the developer may not have thought of directly.  Test frameworks which allow for
generic programming, and runtime test generation are big motivators for me to
use a test framework in my own designs.

\begin{table}[h!]
    \centering
    \begin{tabular}{l | c | c }
        \hline
        & JUnit \autocite{junit} & GTest \autocite{gtest}\\
        \hline \hline
        Language                 & Java  & C++ \\
        Test Generation          & \checkmark \autocite{google_java_tools}
        & \checkmark
        (Section ~\ref{sec:source_code})\\
        Build Server Integration & JUnit XML & Creates
        JUnit compatible XML\\
        IDE Integration          & Eclipse, IntelliJ  &  None \\
        Generic Programming      & Java Generics  &  Templates\\
        Randomization      & Override test runner  &  Test Shuffling\\
        Test Discovery  & Annotations & Macros \\
        Verification Technique & Assert Functions & Assert Macros \\
        License &  Eclipse Public License Version 1.0 & BSD-3-Clause \\
        Integration & Maven & Manual Compilation \\
        \hline
    \end{tabular}
    \caption{Comparison of GTest and JUnit}
    \label{tab:comparison}
\end{table}

\section{Part 2 Demonstration}
The demonstration code verifies a simple selection sort algorithm.  It does so
by generating 10,000 random test vectors. Each vector size is uniformly
distributed between 0 and 1500 elements. Each element in the vector is uniformly
distributed over the range of a 32-bit integer. Using this generated input data,
Google Test instantiates 10,000 instances of the RandomizedTest class, and
executes the tests. 
\singlespace
\subsection{Unit under test}
\label{sec:source_code}
\lstinputlisting[basicstyle=\ttfamily,language=c++]{sort.hpp}
\subsection{Unit Test}
\lstinputlisting[basicstyle=\ttfamily,language=c++]{test.cpp}
\subsection{Unit Test Result}
\lstinputlisting[basicstyle=\ttfamily]{test_output.result}
\subsection{CMake Build Script}
\lstinputlisting[basicstyle=\ttfamily]{CMakeLists.txt}

\clearpage

\printbibliography



\end{document}

